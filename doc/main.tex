\documentclass[12pt]{article}
\usepackage{graphicx}

\usepackage{indentfirst}
\usepackage{polski}
\usepackage{amsmath}
\usepackage{amssymb}
\usepackage{dirtree}
\usepackage{listings}
\usepackage{hyperref}
\hypersetup{
    colorlinks=true,
    linkcolor=blue}
\newtheorem{example}{Przykład}

\title{Dokumentacja projektu zespołowego nr 1}
\author{Anna Ćwiklińska, Krystian Gronkowski, Ihor Malyi}
\date{Marzec 2023}

\begin{document}
\lstset{basicstyle=\ttfamily, columns=fullflexible, upquote=true}
\renewcommand{\lstlistingname}{Listing}

\maketitle

\section{Treść zadania}
Obliczyć $\sqrt{23}$ za pomocą wielomianów interpolacyjnych dla danych z~tabeli:

\begin{table}[h]
\centering
\hypertarget{tabela}{
\begin{tabular}{l|llllllllll}
$x$    & 0 & 1 & 4 & 9 & 16 & 25 & 36 & 49 & 64 & 81 \\ \hline
$f(x)$ & 0 & 1 & 2 & 3 & 4  & 5  & 6  & 7  & 8  & 9  \\
\end{tabular}\\[0.25cm]}
\end{table}


Sprawdzić, jaki podzbiór danych z~tabeli daje najlepsze przybliżenie dokładnej wartości pierwiastka (czyli dla jakiego zestawu tych węzłów wielomian Lagrange’a przebiega najbliżej punktu $(23, \sqrt{23})$).

\section{Teoretyczny opis metody}
W~realizacji naszego projektu, w~celu obliczenia przybliżonej wartości pierwiastka zastosujemy wielomian interpolacyjny Lagrange'a. Wielomian ten będzie konstruowany na podstawie wybranego podzbioru danych z~tabeli, które zostaną wybrane w~sposób optymalny, tak aby przybliżenie było jak najbardziej dokładne.

\subsection{Wielomian interpolacyjny w postaci Lagrange'a}
Niech $n+1$ będzie liczbą węzłów interpolacyjnych, a~$x_0,x_1,\ldots,x_n$ są ich wartościami. Niech $f$ będzie funkcją, którą chcemy interpolować na przedziale $[x_0,x_n]$. Wielomian interpolacyjny $P(x)$ dla $f$ w~węzłach $x_0,x_1,\ldots,x_n$ to unikalny wielomian stopnia co najwyżej $n$, który spełnia $P(x_i) = f(x_i)$ dla $i=0,1,\ldots,n$.

Metoda Lagrange'a polega na wyznaczeniu wielomianu interpolacyjnego za pomocą wzoru:

\begin{equation*}
P(x) = \sum_{i=0}^{n} f(x_i) L_i(x)
\end{equation*}

gdzie $L_i(x)$ to $i$-ty wielomian Lagrange'a, zdefiniowany jako:

\begin{equation*}
L_i(x) = \prod_{j=0, j\neq i}^{n} \frac{x-x_j}{x_i-x_j}
\end{equation*}

\begin{example}

Przyjmijmy, że wybieramy podzbiór danych $${(1,1), (4,2), (9,3), (16,4)}$$ czyli $x_0=1, x_1=4, x_2=9$ i $x_3=16$, oraz odpowiadające im wartości funkcji $f(x)$. W~celu skonstruowania wielomianu interpolacyjnego Lagrange'a dla tego podzbioru danych, należy obliczyć:

\begin{align*}
L_0(x) &= \frac{(x-4)(x-9)(x-16)}{(1-4)(1-9)(1-16)} = -\frac{1}{360}(x^3 - 29x^2 + 244x - 576), \\
L_1(x) &= \frac{(x-1)(x-9)(x-16)}{(4-1)(4-9)(4-16)} = \frac{1}{180}(x^3 - 26x^2 + 169x - 144), \\
L_2(x) &= \frac{(x-1)(x-4)(x-16)}{(9-1)(9-4)(9-16)} = -\frac{1}{280}(x^3 - 21x^2 + 84x - 64), \\
L_3(x) &= \frac{(x-1)(x-4)(x-9)}{(16-1)(16-4)(16-9)} = \frac{1}{1260}(x^3 - 14x^2 + 49x - 36).
\end{align*}

Wielomian interpolacyjny dla wybranego podzbioru danych z~tabeli to \\ $P(x) = f(x_0)L_0(x) + f(x_1)L_1(x) + f(x_2)L_2(x) + f(x_3)L_3(x)$, czyli:

\begin{align*}
P(x) &= 1\cdot\left(-\frac{1}{360}(x^3 - 29x^2 + 244x - 576)\right) \\
&\quad + 2\cdot\frac{1}{180}(x^3 - 26x^2 + 169x - 144) \\
&\quad + 3\cdot\left(-\frac{1}{280}(x^3 - 21x^2 + 84x - 64)\right) \\
&\quad + 4\cdot\frac{1}{1260}(x^3 - 14x^2 + 49x - 36) \\
&= \frac{1}{1260}x^3 - \frac{1}{36}x^2 + \frac{41}{90}x + \frac{4}{7}.
\end{align*}

Aby sprawdzić, jak dobrze wybrany podzbiór danych z~tabeli przybliża dokładną wartość pierwiastka $\sqrt{23}$, możemy porównać wartość wielomianu $P(x)$ dla $x=23$ z~wartością $\sqrt{23}$:

\begin{align*}
P(23) &= \frac{1}{1260}(23)^3 - \frac{1}{36}(23)^2 + \frac{41}{90}(23) + \frac{4}{7} \approx 6,0111111s,
\end{align*}

co jest dość odległym przybliżeniem wartości $\sqrt{23}$.

\end{example}

\section{Opis programu}
Program został napisany w~języku Python w~wersji 3~i~nie korzysta z~żadnych zewnętrznych modułów, co stanowi jego zaletę. Celem programu jest znalezienie wielomianu Lagrange'a, który najlepiej przybliża dane wejściowe dla danego punktu $(23,\sqrt{23})$. Wyniki są porównywane do wartości funkcji \texttt{sqrt(23)} zaimportowanej z biblioteki \texttt{math}.

\subsection{Opis implementacji algorytmu}

\begin{enumerate}
\item W~funkcji \verb|main()| program rozpoczyna się od wczytania danych z~pliku tekstowego o~nazwie \verb|"data.txt"| za pomocą funkcji \verb|import_data()| i~zapisuje je w~zmiennej \verb|data|.
\item Następnie tworzony jest obiekt klasy \verb|Polynomials| na podstawie wczytanych danych.
\item Tworzony jest również obiekt klasy \verb|DataSets| i~przekazywane do niego wczytane dane.
\item W~klasie \verb|DataSets| generowane są wszystkie możliwe kombinacje podzbiorów danych o~różnych rozmiarach za pomocą funkcji \\\verb|generate_subsets()|.
\item Dla każdego wygenerowanego podzbioru danych tworzony jest obiekt klasy \verb|Data|, który reprezentuje dany podzbiór. Klasa \verb|Data| przetwarza ten podzbiór na listę obiektów klasy \verb|Pair|.
\item W~funkcji \verb|main()| wywoływana jest metoda \verb|find_best()| na obiekcie klasy \verb|Polynomials|, która zwraca listę wielomianów posortowaną od najlepszego dopasowania do najgorszego.
\item Tworzony jest obiekt klasy \verb|Polynome|, który reprezentuje najlepszy wielomian Lagrange'a. Do tego obiektu przekazywane są współczynniki oraz dane wejściowe.
\begin{enumerate}
\item Tworzenie obiektu klasy \verb|LagrangeMultipliers|: Program tworzy obiekt klasy \verb|LagrangeMultipliers| za pomocą konstruktora tej klasy, który przyjmuje obiekt klasy \verb|Data| jako argument wejściowy. Wewnątrz konstruktora tworzony jest zestaw wielomianów Lagrange'a za pomocą obiektów klasy \verb|LagrangeMultiplier|, które są przechowywane jako lista w~atrybucie \verb|self.multipliers| w~obiekcie klasy \verb|LagrangeMultipliers|.
\item Tworzenie obiektów klasy \verb|LagrangeMultiplier|: Dla każdego indeksu \verb|i| w~zakresie od~0~do długości danych wejściowych, program tworzy obiekt klasy \verb|LagrangeMultiplier| za pomocą konstruktora tej klasy, który przyjmuje indeks \verb|i| oraz obiekt klasy \verb|Data| jako dane wejściowe. Wewnątrz konstruktora \verb|LagrangeMultiplier| tworzony jest pojedynczy wielomian Lagrange'a za pomocą obiektów klasy \verb|Multiplier|, które są przechowywane jako lista w atrybucie \verb|self.multipliers| w obiekcie klasy \verb|LagrangeMultiplier|.
\item Tworzenie obiektów klasy \verb|Multiplier|: Dla każdej pary danych (\verb|x, y|) w~danych wejściowych dla danego indeksu \verb|i|, program tworzy obiekt klasy \verb|Multiplier| za pomocą konstruktora tej klasy, który przyjmuje wartości \verb|x_k| i~\verb|x_i| jako argumenty wejściowe, gdzie \verb|x_k| jest wartością \verb|x| dla danej pary, a~\verb|x_i| jest wartością \verb|x| dla indeksu, dla którego tworzony jest wielomian Lagrange'a.
\item Obliczanie wartości interpolowanego wielomianu Lagrange'a: Po utworzeniu obiektów klasy \verb|LagrangeMultiplier|, program wywołuje w~klasie \verb|Polynome| metodę \verb|calc(x)| na każdym z~obiektów klasy \verb|Multiplier|, przekazując jej wartość 23, dla której ma zostać obliczona wartość interpolowanego wielomianu Lagrange'a.
\end{enumerate}
\item Na koniec program wypisuje na ekranie najlepszy wielomian Lagrange'a wraz z~jego współczynnikami, na podstawie obiektu klasy \verb|Polynomial| oraz generuje raport w~pliku \verb|best.tex|.
\item Program kończy swoje działanie.
\end{enumerate}

\section{Instrukcja użytkowania}
Do prawidłowego działania programu wymagany jest zainstalowany interpreter języka Python~3. Należy również upewnić się, że struktura plików programu jest kompletna, a~w~szczegółności, że plik z danymi wejściowymi (opisany szczegółowo poniżej) znajduje się we właściwym miejscu.\\

Aby uruchomić program, należy:
\begin{enumerate}
    \item Otworzyć terminal lub wiersz polecenia na swoim komputerze.
    \item Przejść do katalogu, w którym znajduje się plik \texttt{main.py}.
    \item (Opcjonalnie) Zmienić dane wejściowe w pliku \texttt{data.txt}
    \item Uruchomić program komputerowy za pomocą: \texttt{python main.py} (lub \texttt{py main.py})
\end{enumerate}
Jeśli wszystko się powiodło, program wczyta dane z pliku i wykona się. W terminalu powinny wyświetlić się dane wyjściowe (opisane poniżej). W folderze \texttt{reports} powstanie nowy plik w formacie \texttt{.tex}, którego tytuł będzie połączeniem słowa \texttt{report} oraz daty i godziny wykonania programu.
\subsection{Dane wejściowe}
Program oczekuje na dane wejściowe dostarczone w~pliku \texttt{data.txt}, który powinien znajdować się w~tym samym katalogu co program. Domyślnie w~pliku znajdują się pełne dane z~\hyperlink{tabela}{tabeli} z~punktu 1. Poprzez modyfikację pliku \texttt{dane.txt} użytkownik może zmieniać zakres istniejących danych wejściowych oraz dodawać nowe. 

Dane wejściowe powinny być podane w~formacie \texttt{x y}, gdzie~\texttt{x} i~\texttt{y} są odpowiednio współrzędnymi $(x,y)$ danego węzła i~są oddzielone spacją. Przykładowy format danych wejściowych przedstawia się następująco: 
\begin{lstlisting}[language = Python]
                                1 1
                                4 2
                                9 3
                                16 4
                                25 5
\end{lstlisting}
\subsection{Walidacja danych wejściowych}
Za proces importu danych do programu oraz ich walidację odpowiada funkcja \texttt{import\_data} zlokalizowana w~pliku \texttt{getData.py}.

Funkcja sprawdza każdą linijkę, czy zawiera dwie liczby całkowite oddzielone spacją, a~następnie dodaje te liczby do listy \texttt{data}. 

Jeśli program napotka pustą linijkę, lub spacji w linijce jest więcej niż jedna, nie wpływa to na działanie programu. 

Jeśli plik nie istnieje lub linijka nie zawiera dwóch liczb, funkcja wypisuje odpowiedni komunikat o~błędzie i~program kończy działanie. 

W~przypadku niepowodzenia konwersji wartości na liczby całkowite, również wypisuje komunikat o~błędzie i~kończy działanie.

Gdy wszystko się powiedzie, funkcja na końcu zwraca listę z~danymi.

\subsection{Dane wyjściowe}
Program generuje raport w~formacie \texttt{.tex} z~obliczonymi wartościami najlepiej dopasowanego wielomianu. Zwraca również wyniki w~postaci tekstu w~terminalu – zestawu danych w kolejności od najlepszego do najgorszego dopasowania.

Wynik w~terminalu zostanie zwrócony w~postaci: $$(x_1, y_1), (x_2, y_2)\ldots (x_n, y_n) \rightarrow wynik$$

Na przykład:\\[0.15cm]
\texttt{(1, 1), (4, 2), (64, 8) -> 6.785185185185185}\\

gdzie:\\[0.15cm]\texttt{(1, 1), (4, 2), (64, 8)} to dane wejściowe, które najlepiej przybliżają funkcję \texttt{sqrt},\\[0.15cm]\texttt{6.785185185185185} to wynik obliczenia wielomianu Lagrange'a.

\subsection{Przykładowe wyniki programu}

\section{Struktura plików}
\dirtree{%
    .1 program.
    .2 lagrange.
    .3 data.py.
    .3 data.txt.
    .3 getData.py.
    .3 lagrange\_multiplier.py.
    .3 lagrange\_multipliers.py.
    .3 main.py.
    .3 multiplier.py.
    .3 polynome.py.
    .3 polynomials.py.
    .3 report.py.
    .3 tests.py.
    .2 reports.
    .3 \ldots.
}
\subsection{Opis struktury plików}
\begin{itemize}
    \item Folder \verb|program| – folder nadrzędny, zawiera wszystkie pliki programu
    \begin{itemize}
        \item Folder \verb|lagrange| – zawiera pliki odpowiadające za funkcjonowanie programu
        \begin{itemize}
            \item Plik \verb|data.py| zawiera definicje klas \verb|Pair|, \verb|DataSets| i \verb|Data|.
            \item \verb|data.txt| to plik tekstowy służący do przechowywania danych wejściowych programu. Ten plik należy zmodyfikować, chcąc użyć innych danych niż domyślne.
            \item Plik \verb|getData.py| zawiera funkcję importującą dane z pliku oraz sprawdzającą ich poprawność.
            \item Plik \verb|lagrange_multiplier.py| zawiera klasę reprezentującą wielomian Lagrange'a dla jednego punktu danych. Oblicza wartość wielomianu Lagrange'a dla danego $x$.
            \item Plik \verb|lagrange_multipliers.py| zawiera klasę reprezentującą wielomiany Lagrange'a dla całego zbioru danych
            \item Plik \verb|main.py| zawiera główną funkcję programu.
            \item Plik \verb|multiplier.py| zawiera klasę, która reprezentuje obiekt do obliczania wartości na podstawie podanego wzoru matematycznego.
            \item Plik \verb|polynome.py| zawiera klasę \verb|Polynome|, która reprezentuje wielomian interpolacyjny Lagrange'a dla danego zbioru danych.
            \item Plik \verb|polynomials.py| zawiera klasę reprezentującą zbiór wielomianów interpolacyjnych Lagrange'a dla różnych zbiorów danych.
            \item Plik \verb|report.py| odpowiada za generowanie raportów dla najlepszych dopasowań wielomianów.
            \item Plik \verb|tests.py| zawiera kod do testowania funkcji w innych modułach
        \end{itemize}
        \item Folder \verb|reports| – zawiera raporty w~formacie .tex wygenerowane przez kolejne uruchomienia programu.
    \end{itemize}
\end{itemize}

\section{Raport z~demonstracji}

\section{Wnioski i~interpretacja wyników}

\section*{Bibliografia}
Danuta Jaruszewska-Walczak - wykład Algorytmy numeryczne

\end{document}
