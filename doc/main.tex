\documentclass[12pt]{article}
\usepackage{graphicx}

\usepackage{indentfirst}
\usepackage{polski}
\usepackage{amsmath}
\usepackage{amssymb}
\usepackage{dirtree}
\usepackage{listings}
\usepackage{hyperref}
\hypersetup{
    colorlinks=true,
    linkcolor=blue}
\newtheorem{example}{Przykład}

\usepackage{graphicx}
\graphicspath{ {./} }
\usepackage{xcolor}
\usepackage{listings}
\usepackage{tikz}
\usetikzlibrary{matrix}
\graphicspath{ {./} }
\definecolor{codegreen}{rgb}{0,0.6,0}
\definecolor{codegray}{rgb}{0.5,0.5,0.5}
\definecolor{codepurple}{rgb}{0.58,0,0.82}
\definecolor{backcolour}{rgb}{0.95,0.95,0.92}
\definecolor{commentsColor}{rgb}{0.497495, 0.497587, 0.497464}
\definecolor{keywordsColor}{rgb}{0.000000, 0.000000, 0.635294}
\definecolor{stringColor}{rgb}{0.558215, 0.000000, 0.135316}
\definecolor{backcolor}{rgb}{0.95,0.95,0.92}
\lstset{ %
  backgroundcolor=\color{backcolor},   % choose the background color; you must add \usepackage{color} or \usepackage{xcolor}
  basicstyle=\footnotesize,        % the size of the fonts that are used for the code
  breakatwhitespace=false,         % sets if automatic breaks should only happen at whitespace
  breaklines=true,                 % sets automatic line breaking
  captionpos=b,                    % sets the caption-position to bottom
  commentstyle=\color{commentsColor}\textit,    % comment style
  deletekeywords={...},            % if you want to delete keywords from the given language
  escapeinside={\%*}{*)},          % if you want to add LaTeX within your code
  extendedchars=true,              % lets you use non-ASCII characters; for 8-bits encodings only, does not work with UTF-8
  frame=tb,	                   	   % adds a frame around the code
  keepspaces=true,                 % keeps spaces in text, useful for keeping indentation of code (possibly needs columns=flexible)
  keywordstyle=\color{keywordsColor}\bfseries,       % keyword style
  language=Python,                 % the language of the code (can be overrided per snippet)
  otherkeywords={*,...},           % if you want to add more keywords to the set
  numbers=left,                    % where to put the line-numbers; possible values are (none, left, right)
  numbersep=5pt,                   % how far the line-numbers are from the code
  numberstyle=\tiny\color{commentsColor}, % the style that is used for the line-numbers
  rulecolor=\color{black},         % if not set, the frame-color may be changed on line-breaks within not-black text (e.g. comments (green here))
  showspaces=false,                % show spaces everywhere adding particular underscores; it overrides 'showstringspaces'
  showstringspaces=false,          % underline spaces within strings only
  showtabs=false,                  % show tabs within strings adding particular underscores
  stepnumber=1,                    % the step between two line-numbers. If it's 1, each line will be numbered
  stringstyle=\color{stringColor}, % string literal style
  tabsize=2,	                   % sets default tabsize to 2 spaces
  title=\lstname,                  % show the filename of files included with \lstinputlisting; also try caption instead of title
  columns=fixed                    % Using fixed column width (for e.g. nice alignment)
}


\title{Dokumentacja projektu zespołowego nr 1}
\author{Anna Ćwiklińska, Krystian Gronkowski, Ihor Malyi}
\date{Marzec 2023}

\begin{document}
\lstset{basicstyle=\ttfamily, columns=fullflexible, upquote=true}
\renewcommand{\lstlistingname}{Listing}

\maketitle

\section{Treść zadania}
Obliczyć $\sqrt{23}$ za pomocą wielomianów interpolacyjnych dla danych z~tabeli:

\begin{table}[h]
\centering
\hypertarget{tabela}{
\begin{tabular}{l|llllllllll}
$x$    & 0 & 1 & 4 & 9 & 16 & 25 & 36 & 49 & 64 & 81 \\ \hline
$f(x)$ & 0 & 1 & 2 & 3 & 4  & 5  & 6  & 7  & 8  & 9  \\
\end{tabular}\\[0.25cm]}
\end{table}


Sprawdzić, jaki podzbiór danych z~tabeli daje najlepsze przybliżenie dokładnej wartości pierwiastka (czyli dla jakiego zestawu tych węzłów wielomian Lagrange’a przebiega najbliżej punktu $(23, \sqrt{23})$).

\section{Teoretyczny opis metody}
W~realizacji naszego projektu, w~celu obliczenia przybliżonej wartości pierwiastka zastosujemy wielomian interpolacyjny Lagrange'a. Wielomian ten będzie konstruowany na podstawie wybranego podzbioru danych z~tabeli, które zostaną wybrane w~sposób optymalny, tak aby przybliżenie było jak najbardziej dokładne.

\subsection{Wielomian interpolacyjny w postaci Lagrange'a}
Niech $n+1$ będzie liczbą węzłów interpolacyjnych, a~$x_0,x_1,\ldots,x_n$ są ich wartościami. Niech $f$ będzie funkcją, którą chcemy interpolować na przedziale $[x_0,x_n]$. Wielomian interpolacyjny $P(x)$ dla $f$ w~węzłach $x_0,x_1,\ldots,x_n$ to unikalny wielomian stopnia co najwyżej $n$, który spełnia $P(x_i) = f(x_i)$ dla $i=0,1,\ldots,n$.

Metoda Lagrange'a polega na wyznaczeniu wielomianu interpolacyjnego za pomocą wzoru:

\begin{equation*}
P(x) = \sum_{i=0}^{n} f(x_i) L_i(x)
\end{equation*}

gdzie $L_i(x)$ to $i$-ty wielomian Lagrange'a, zdefiniowany jako:

\begin{equation*}
L_i(x) = \prod_{j=0, j\neq i}^{n} \frac{x-x_j}{x_i-x_j}
\end{equation*}

\begin{example}

Przyjmijmy, że wybieramy podzbiór danych $${(1,1), (4,2), (9,3), (16,4)}$$ czyli $x_0=1, x_1=4, x_2=9$ i $x_3=16$, oraz odpowiadające im wartości funkcji $f(x)$. W~celu skonstruowania wielomianu interpolacyjnego Lagrange'a dla tego podzbioru danych, należy obliczyć:

\begin{align*}
L_0(x) &= \frac{(x-4)(x-9)(x-16)}{(1-4)(1-9)(1-16)} = -\frac{1}{360}(x^3 - 29x^2 + 244x - 576), \\
L_1(x) &= \frac{(x-1)(x-9)(x-16)}{(4-1)(4-9)(4-16)} = \frac{1}{180}(x^3 - 26x^2 + 169x - 144), \\
L_2(x) &= \frac{(x-1)(x-4)(x-16)}{(9-1)(9-4)(9-16)} = -\frac{1}{280}(x^3 - 21x^2 + 84x - 64), \\
L_3(x) &= \frac{(x-1)(x-4)(x-9)}{(16-1)(16-4)(16-9)} = \frac{1}{1260}(x^3 - 14x^2 + 49x - 36).
\end{align*}

Wielomian interpolacyjny dla wybranego podzbioru danych z~tabeli to \\ $P(x) = f(x_0)L_0(x) + f(x_1)L_1(x) + f(x_2)L_2(x) + f(x_3)L_3(x)$, czyli:

\begin{align*}
P(x) &= 1\cdot\left(-\frac{1}{360}(x^3 - 29x^2 + 244x - 576)\right) \\
&\quad + 2\cdot\frac{1}{180}(x^3 - 26x^2 + 169x - 144) \\
&\quad + 3\cdot\left(-\frac{1}{280}(x^3 - 21x^2 + 84x - 64)\right) \\
&\quad + 4\cdot\frac{1}{1260}(x^3 - 14x^2 + 49x - 36) \\
&= \frac{1}{1260}x^3 - \frac{1}{36}x^2 + \frac{41}{90}x + \frac{4}{7}.
\end{align*}

Aby sprawdzić, jak dobrze wybrany podzbiór danych z~tabeli przybliża dokładną wartość pierwiastka $\sqrt{23}$, możemy porównać wartość wielomianu $P(x)$ dla $x=23$ z~wartością $\sqrt{23}$:

\begin{align*}
P(23) &= \frac{1}{1260}(23)^3 - \frac{1}{36}(23)^2 + \frac{41}{90}(23) + \frac{4}{7} \approx 6,0111111s,
\end{align*}

co jest dość odległym przybliżeniem wartości $\sqrt{23}$.

\end{example}

\section{Opis programu}
Program został napisany w~języku Python w~wersji 3~i~nie korzysta z~żadnych zewnętrznych modułów, co stanowi jego zaletę. Celem programu jest znalezienie wielomianu Lagrange'a, który najlepiej przybliża dane wejściowe dla danego punktu $(23,\sqrt{23})$. Wyniki są porównywane do wartości funkcji \texttt{sqrt(23)} zaimportowanej z biblioteki \texttt{math}.

\subsection{Opis implementacji algorytmu}

\begin{enumerate}
\item W~funkcji \verb|main()| program rozpoczyna się od wczytania danych z~pliku tekstowego o~nazwie \verb|"data.txt"| za pomocą funkcji \verb|import_data()| i~zapisuje je w~zmiennej \verb|data|.
\item Następnie tworzony jest obiekt klasy \verb|Polynomials| na podstawie wczytanych danych.
\item Tworzony jest również obiekt klasy \verb|DataSets| i~przekazywane do niego wczytane dane.
\item W~klasie \verb|DataSets| generowane są wszystkie możliwe kombinacje podzbiorów danych o~różnych rozmiarach za pomocą funkcji \\\verb|generate_subsets()|.
\item Dla każdego wygenerowanego podzbioru danych tworzony jest obiekt klasy \verb|Data|, który reprezentuje dany podzbiór. Klasa \verb|Data| przetwarza ten podzbiór na listę obiektów klasy \verb|Pair|.
\item W~funkcji \verb|main()| wywoływana jest metoda \verb|find_best()| na obiekcie klasy \verb|Polynomials|, która zwraca listę wielomianów posortowaną od najlepszego dopasowania do najgorszego.
\item Tworzony jest obiekt klasy \verb|Polynome|, który reprezentuje najlepszy wielomian Lagrange'a. Do tego obiektu przekazywane są współczynniki oraz dane wejściowe.
\begin{enumerate}
\item Tworzenie obiektu klasy \verb|LagrangeMultipliers|: Program tworzy obiekt klasy \verb|LagrangeMultipliers| za pomocą konstruktora tej klasy, który przyjmuje obiekt klasy \verb|Data| jako argument wejściowy. Wewnątrz konstruktora tworzony jest zestaw wielomianów Lagrange'a za pomocą obiektów klasy \verb|LagrangeMultiplier|, które są przechowywane jako lista w~atrybucie \verb|self.multipliers| w~obiekcie klasy \verb|LagrangeMultipliers|.
\item Tworzenie obiektów klasy \verb|LagrangeMultiplier|: Dla każdego indeksu \verb|i| w~zakresie od~0~do długości danych wejściowych, program tworzy obiekt klasy \verb|LagrangeMultiplier| za pomocą konstruktora tej klasy, który przyjmuje indeks \verb|i| oraz obiekt klasy \verb|Data| jako dane wejściowe. Wewnątrz konstruktora \verb|LagrangeMultiplier| tworzony jest pojedynczy wielomian Lagrange'a za pomocą obiektów klasy \verb|Multiplier|, które są przechowywane jako lista w atrybucie \verb|self.multipliers| w obiekcie klasy \verb|LagrangeMultiplier|.
\item Tworzenie obiektów klasy \verb|Multiplier|: Dla każdej pary danych (\verb|x, y|) w~danych wejściowych dla danego indeksu \verb|i|, program tworzy obiekt klasy \verb|Multiplier| za pomocą konstruktora tej klasy, który przyjmuje wartości \verb|x_k| i~\verb|x_i| jako argumenty wejściowe, gdzie \verb|x_k| jest wartością \verb|x| dla danej pary, a~\verb|x_i| jest wartością \verb|x| dla indeksu, dla którego tworzony jest wielomian Lagrange'a.
\item Obliczanie wartości interpolowanego wielomianu Lagrange'a: Po utworzeniu obiektów klasy \verb|LagrangeMultiplier|, program wywołuje w~klasie \verb|Polynome| metodę \verb|calc(x)| na każdym z~obiektów klasy \verb|Multiplier|, przekazując jej wartość 23, dla której ma zostać obliczona wartość interpolowanego wielomianu Lagrange'a.
\end{enumerate}
\item Na koniec program wypisuje na ekranie najlepszy wielomian Lagrange'a wraz z~jego współczynnikami, na podstawie obiektu klasy \verb|Polynomial| oraz generuje raport w~pliku \verb|best.tex|.
\item Program kończy swoje działanie.
\end{enumerate}

\section{Instrukcja użytkowania}
Do prawidłowego działania programu wymagany jest zainstalowany interpreter języka Python~3. Należy również upewnić się, że struktura plików programu jest kompletna, a~w~szczegółności, że plik z danymi wejściowymi (opisany szczegółowo poniżej) znajduje się we właściwym miejscu.\\

Aby uruchomić program, należy:
\begin{enumerate}
    \item Otworzyć terminal lub wiersz polecenia na swoim komputerze.
    \item Przejść do katalogu, w którym znajduje się plik \texttt{main.py}.
    \item (Opcjonalnie) Zmienić dane wejściowe w pliku \texttt{data.txt}
    \item Uruchomić program komputerowy za pomocą: \texttt{python main.py} (lub \texttt{py main.py})
\end{enumerate}
Jeśli wszystko się powiodło, program wczyta dane z pliku i wykona się. W terminalu powinny wyświetlić się dane wyjściowe (opisane poniżej). W folderze \texttt{reports} powstanie nowy plik w formacie \texttt{.tex}, którego tytuł będzie połączeniem słowa \texttt{report} oraz daty i godziny wykonania programu.
\subsection{Dane wejściowe}
Program oczekuje na dane wejściowe dostarczone w~pliku \texttt{data.txt}, który powinien znajdować się w~tym samym katalogu co program. Domyślnie w~pliku znajdują się pełne dane z~\hyperlink{tabela}{tabeli} z~punktu 1. Poprzez modyfikację pliku \texttt{dane.txt} użytkownik może zmieniać zakres istniejących danych wejściowych oraz dodawać nowe. 

Dane wejściowe powinny być podane w~formacie \texttt{x y}, gdzie~\texttt{x} i~\texttt{y} są odpowiednio współrzędnymi $(x,y)$ danego węzła i~są oddzielone spacją. Przykładowy format danych wejściowych przedstawia się następująco: 
\begin{lstlisting}[language = Python]
                                1 1
                                4 2
                                9 3
                                16 4
                                25 5
\end{lstlisting}
\subsection{Walidacja danych wejściowych}
Za proces importu danych do programu oraz ich walidację odpowiada funkcja \texttt{import\_data} zlokalizowana w~pliku \texttt{getData.py}.

Funkcja sprawdza każdą linijkę, czy zawiera dwie liczby całkowite oddzielone spacją, a~następnie dodaje te liczby do listy \texttt{data}. 

Jeśli program napotka pustą linijkę, lub spacji w linijce jest więcej niż jedna, nie wpływa to na działanie programu. 

Jeśli plik nie istnieje lub linijka nie zawiera dwóch liczb, funkcja wypisuje odpowiedni komunikat o~błędzie i~program kończy działanie. 

W~przypadku niepowodzenia konwersji wartości na liczby całkowite, również wypisuje komunikat o~błędzie i~kończy działanie.

Gdy wszystko się powiedzie, funkcja na końcu zwraca listę z~danymi.

\subsection{Dane wyjściowe}
Program generuje raport w~formacie \texttt{.tex} z~obliczonymi wartościami najlepiej dopasowanego wielomianu. Zwraca również wyniki w~postaci tekstu w~terminalu – zestawu danych w kolejności od najlepszego do najgorszego dopasowania.

Wynik w~terminalu zostanie zwrócony w~postaci: $$(x_1, y_1), (x_2, y_2)\ldots (x_n, y_n) \rightarrow wynik$$

Na przykład:\\[0.15cm]
\texttt{(1, 1), (4, 2), (64, 8) -> 6.785185185185185}\\

gdzie:\\[0.15cm]\texttt{(1, 1), (4, 2), (64, 8)} to dane wejściowe, które najlepiej przybliżają funkcję \texttt{sqrt},\\[0.15cm]\texttt{6.785185185185185} to wynik obliczenia wielomianu Lagrange'a.

\subsection{Przykładowe wyniki programu}

\section{Struktura plików}
\dirtree{%
    .1 program.
    .2 lagrange.
    .3 data.py.
    .3 data.txt.
    .3 getData.py.
    .3 lagrange\_multiplier.py.
    .3 lagrange\_multipliers.py.
    .3 main.py.
    .3 multiplier.py.
    .3 polynome.py.
    .3 polynomials.py.
    .3 report.py.
    .3 tests.py.
    .2 reports.
    .3 \ldots.
}
\subsection{Opis struktury plików}
\begin{itemize}
    \item Folder \verb|program| – folder nadrzędny, zawiera wszystkie pliki programu
    \begin{itemize}
        \item Folder \verb|lagrange| – zawiera pliki odpowiadające za funkcjonowanie programu
        \begin{itemize}
            \item Plik \verb|data.py| zawiera definicje klas \verb|Pair|, \verb|DataSets| i \verb|Data|.
            \item \verb|data.txt| to plik tekstowy służący do przechowywania danych wejściowych programu. Ten plik należy zmodyfikować, chcąc użyć innych danych niż domyślne.
            \item Plik \verb|getData.py| zawiera funkcję importującą dane z pliku oraz sprawdzającą ich poprawność.
            \item Plik \verb|lagrange_multiplier.py| zawiera klasę reprezentującą wielomian Lagrange'a dla jednego punktu danych. Oblicza wartość wielomianu Lagrange'a dla danego $x$.
            \item Plik \verb|lagrange_multipliers.py| zawiera klasę reprezentującą wielomiany Lagrange'a dla całego zbioru danych
            \item Plik \verb|main.py| zawiera główną funkcję programu.
            \item Plik \verb|multiplier.py| zawiera klasę, która reprezentuje obiekt do obliczania wartości na podstawie podanego wzoru matematycznego.
            \item Plik \verb|polynome.py| zawiera klasę \verb|Polynome|, która reprezentuje wielomian interpolacyjny Lagrange'a dla danego zbioru danych.
            \item Plik \verb|polynomials.py| zawiera klasę reprezentującą zbiór wielomianów interpolacyjnych Lagrange'a dla różnych zbiorów danych.
            \item Plik \verb|report.py| odpowiada za generowanie raportów dla najlepszych dopasowań wielomianów.
            \item Plik \verb|tests.py| zawiera kod do testowania funkcji w innych modułach
        \end{itemize}
        \item Folder \verb|reports| – zawiera raporty w~formacie .tex wygenerowane przez kolejne uruchomienia programu.
    \end{itemize}
\end{itemize}

\section{Raport z~demonstracji}

\section{Wnioski i~interpretacja wyników}

\section{Kod programu}
Main.py
\begin{lstlisting}[frame=single]
import math

from polynomials import Polynomials
from getData import import_data
from math import sqrt
from plot import Plot


def main():
    p = Polynomials(import_data())
    best = p.find_best()
    print(f"sqrt(23) = {math.sqrt(23)}")
    for b in best:
        print(b)
    best[-1].report.generate()
    best[-1].report.save()


if __name__ == "__main__":
    main()

\end{lstlisting}

getData.py
\begin{lstlisting}[frame=single]
def import_data():
    data = []
    try:
        with open("./data.txt") as f:
            for line in f:
                line = line.strip()
                if not line:
                    continue
                values = line.split()
                if len(values) != 2:
                    print("Błąd: wiersz nie zawiera dwóch liczb oddzielonych spacją!")
                    exit()
                try:
                    x, y = int(values[0]), int(values[1])
                except ValueError:
                    print("Błąd: nie można przekonwertować wartości na liczby całkowite!")
                    exit()
                data.append([x, y])
    except FileNotFoundError:
        print("Błąd: plik nie istnieje!")
        exit()
    return data
\end{lstlisting}

Lagrangemultiplier.py
\begin{lstlisting}[frame=single]
from data import Data
from multiplier import Multilplier


class LagrangeMultiplier:
    def __init__(self, i: int, data: Data):
        self.multipliers = []
        self.i = i
        for pair in data.get_data(i):
            x_i = data.data[i].x
            self.multipliers.append(Multilplier(pair.x, x_i))

    def calc(self, x: int):
        res = 1
        for multiplier in self.multipliers:
            res = res * multiplier.calc(x)
        return res

    def __to_str(self):
        mult = list(map(lambda m: str(m), self.multipliers))
        res = "\cdot".join(mult)
        res = f"l_{self.i}(x)={res}"
        res = res.replace('--', '+')
        return res

    def __str__(self):
        return self.__to_str()

    def __repr__(self):
        return self.__to_str()
\end{lstlisting}

Lagrangemultipliers.py
\begin{lstlisting}[frame=single]
from data import Data
from data import Data
from lagrange_multiplier import LagrangeMultiplier


class LagrangeMultipliers:
    def __init__(self, data: Data):
        self.multipliers = []
        for i in range(len(data.data)):
            self.multipliers.append(LagrangeMultiplier(i, data))
\end{lstlisting}

Multiplier.py
\begin{lstlisting}[frame=single]
class Multilplier:
    def __init__(self, x_k, x_i):
        self.x_k = x_k
        self.x_i = x_i
        self.value = None

    def calc(self, x):
        self.value = (x - self.x_k) / (self.x_i - self.x_k)
        return self.value

    def __str__(self):
        return f"\\frac{{x-{self.x_k}}}{{ {self.x_i}-{self.x_k}}}"

    def __repr__(self):
        return f"\\frac{{x-{self.x_k}}}{{ {self.x_i}-{self.x_k}}}"
\end{lstlisting}

Polynomials.py
\begin{lstlisting}[frame=single]
from polynome import Polynome
from math import sqrt
from data import DataSets


class Polynomials:
    """
    A class that represents a collection of polynomial objects that can be used to find the best polynomial
    approximation for a given set of data.
    """

    def __init__(self, datasets: list):
        """
        Initializes an object of the Polynomials class with a single parameter - datasets.
        :param datasets: A DataSets object containing a list of data points.
        """
        self.datasets = DataSets(datasets)

    def find_best(self):
        """
        Computes a set of polynomial objects for each dataset in the Polynomials object and returns the best
        polynomial approximation for the dataset.
        :return: A polynomial object that best approximates the dataset.
        """
        polynomials = []
        for data in self.datasets:
            p = Polynome(data)
            polynomials.append(p)
        best = sorted(polynomials, key=lambda p: abs(p.value - sqrt(23)))
        return best

\end{lstlisting}
Polynome.py
\begin{lstlisting}[frame=single]
from data import Data
from report import Report
from lagrange_multipliers import LagrangeMultipliers
from common_formula import CommonPolynomeFormula
from plot import Plot

class Polynome:
    def __init__(self, data: Data):
        self.data = data
        self.report = Report(self)
        self.common_formula = CommonPolynomeFormula(self.data).policz()
        self.multipliers = LagrangeMultipliers(data)
        self.plot = Plot(self)

    def format_common_formula(self):
        formatted_terms = []
        for term in self.common_formula.split():
            if 'e' in term:
                coeff, exp = term.split('e')
                formatted_term = f'{coeff} \\times 10^{{{exp}}}'
            else:
                formatted_term = term
            formatted_terms.append(formatted_term)

        formatted_formula = ''
        for i, term in enumerate(formatted_terms):
            formatted_formula += term
            if formatted_formula.endswith(('+', '-')):
                formatted_formula += '\\\\\\indent'
        return formatted_formula

    def calc(self, x):
        tmp = 0
        for multiplier, data_ in zip(self.multipliers.multipliers, self.data.data):
            tmp = tmp + multiplier.calc(x) * data_.y
        return tmp

    @property
    def value(self):
        return self.calc(23)

    def __str__(self):
        return f"{self.data} -> {self.value}"


    def __repr__(self):
        return f"{self.data} -> {self.value}"
\end{lstlisting}
Report.py
\begin{lstlisting}[frame=single]
import os
from datetime import datetime


class Report:
    def __init__(self, polynome):
        self.polynome = polynome

    def save(self):
        now = datetime.now()
        current_datetime = now.strftime("%Y%m%d_%H%M%S")
        filename = f"report_{current_datetime}.tex"
        file_path = os.path.join('../reports/', filename)
        with open(file_path, 'w') as f:
            f.write(self.generate())

    def __generate_multipliers(self):
        multipliers_view = []

        for multiplier in self.polynome.multipliers.multipliers:
            multipliers_view.append(f"${multiplier}$")
        multipliers_view = ' \\\ '.join(multipliers_view)
        return multipliers_view

    def __generate_p(self):
        res = "$P(x)="
        terms = []
        i = 0
        for x_y in self.polynome.data.data:
            terms.append(f"{x_y.y}\cdot l_{i}")
            i += 1
        res += "+".join(terms)
        res += "$"
        return res

    def generate(self):
        self.__generate_multipliers()
        self.polynome.plot.plot()
        content = f"""\documentclass{{article}}
        \\usepackage{{polski}}
        \\usepackage{{graphicx}}
        \\begin{{document}}
        
        Obliczenie najlepszego przybliżenia \sqrt(23):\\\\[0.25cm]
        \\begin{{align*}}
        {self.__generate_multipliers()} \\\\[0.25cm]
        {self.__generate_p()} \\\\[0.25cm]
        \\end{{align*}}\\\\[0.25cm]
        Wzór ogólny wielomianu:\\\\[0.25cm]
        \\begin{{split}}
        $P(x) =${self.polynome.format_common_formula()}
        \\end{{split}}
        
        \includegraphics{{{self.polynome.plot.filename}}}
        
        \\end{{document}}"""
        return content

    def __str__(self):
        return self.generate()
\end{lstlisting}
Plot.py
\begin{lstlisting}[frame=single]
import os
# jezeli nie dziala
# pip3 install matplotlib
# jezeli nie dziala pip3 install matplotlib na sigmie
# musisz utworzyc virtualenv w pycharm i instalowac w utworzony virtualenv
import math

import matplotlib.pyplot as plt
import datetime


class Plot:
    def __init__(self, polynome):
        self.polynome = polynome
        self.__filename = None

    @property
    def filename(self):
        if not self.__filename:
            self.__filename = str(datetime.datetime.now().timestamp()) + '.png'
        return self.__filename

    def calc_points(self, func):
        end = 25
        x = []
        y = []
        i = 22
        while i < end:
            x.append(i)
            y.append(func(i))
            i += 0.0001
        return x, y

    def plot(self):
        for f in [self.polynome.calc, math.sqrt]:
            x, y = self.calc_points(f)
            plt.plot(x, y, label=f)
            plt.ylabel('Wartosc')
            plt.xlabel('x')

        with open('../reports/' + self.filename, 'wb') as f:
            plt.savefig(f)
\end{lstlisting}

\section*{Bibliografia}
Danuta Jaruszewska-Walczak - wykład Algorytmy numeryczne

\end{document}